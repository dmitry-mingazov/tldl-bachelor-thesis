\chapter{User Manual}
\section{Software documentation - Frontend}
To create a new widget you must extend the \verb|BaseWidget| and \verb|BaseWidgetComponent| [\ref{subsec:base}] classes respectively for the \verb|Widget| and \verb|WidgetComponent|[\ref{chap:widget}] classes.\\
Then you must decorate your \verb|Widget| with the \verb|WidgetDecorator| which takes 5 fields to set the configuration.
\begin{lstlisting}[caption={PieChartWidget example}, style=javaScriptCode]
@WidgetDecorator({
    endpoint: 'tuple', 
    type: WidgetTypeEnum.MOBILITY,
    sourceable: true,
    refreshableInfo: {
        refreshable: true,
        defaultRefreshRate: RefreshRateEnum.sec_5
    },
    adapter: PieChartAdapter,
})
export class PieChartWidget extends BaseWidget {

...

}
\end{lstlisting}
And this is the \textbf{structure} of the options passed to the decorator [\ref{subsec:widgetDecorator}]:
\begin{lstlisting}[caption={Structure of WidgetOptions}, style=javaScriptCode]
export interface WidgetOptions {
    endpoint: string;
    type: WidgetTypeEnum;
    sourceable?: boolean;
    refreshableInfo?: RefreshableInfo; 
    adapter?: any;
}

export interface RefreshableInfo {
    refreshable: boolean;
    defaultRefreshRate?: RefreshRateEnum;
}
\end{lstlisting}
\begin{itemize}
    \item \verb|endpoint|: indicates which endpoint the widget has to use to take data from.
    \item \verb|type|: indicates the widget type (currently IT/MOBILITY).
    \item \verb|sourceable| (optional): indicates whether or not the widget has a selectable source parameter.
    \item \verb|refreshableInfo| (optional): indicates whether or not the widget has a refresh parameter and can be used to set the default refresh rate.
    \item \verb|adapter| (optional): indicates the adapter class used to transform received data into a more useful form.
\end{itemize}
You can create an \verb|Adapter| class implementing the \verb|Adapter| functional interface and the \verb|adapt| method.
\begin{lstlisting}[caption={Adapter interface}, style=javaScriptCode]
export interface Adapter {
    adapt(dto: any): any;
}
\end{lstlisting}
Lastly inside the \verb|WidgetComponent| you must implement the \verb|getDataCallback| method where \verb|value| is the data received from the backend.  This method will be called everytime the widget receives data.\\
\textbf{Note}: \verb|value| will be adapted data if an \verb|Adapter| for that class exists.
\begin{lstlisting}[caption={PieChartComponent example}, style=javaScriptCode]
@Component({
    selector: 'app-pie-chart-widget',
    templateUrl: './pie-chart.component.html',
    encapsulation: ViewEncapsulation.None,
    changeDetection: ChangeDetectionStrategy.OnPush
})
export class PieChartComponent extends BaseWidgetComponent<PieChartWidget> {

    public data: TupleData[] = [];
  
    ...

    protected getDataCallback(value: any): void {
        this.data = value;
    }
}
\end{lstlisting}

\section{Software documentation - Backend}
To add a new widget endpoint you need to follow this steps:
\begin{itemize}
    \item Implement the data access layer.
    \item Add a \verb|{sourcecode}.json| file in \verb|DiscoveryConfiguration/parameters| [\ref{subsec:source}].
    \begin{lstlisting}[caption={{sourcecode}.json example}, style=javaScriptCode]
    {
        "metadata": {
            "meta": {
                "fieldName": "label"
            },
        "properties": {
            "color": [
                {
                    "field": "Passed",
                    "value": "green"
                },
                {
                    "field": "Refused",
                    "value": "red"
                }
            ],
            "field": [
                {
                    "field": "Passed",
                    "value": "ok"
                },
                {
                    "field": "Refused",
                    "value": "ko"
                }
            ]
        }
        },
        "filters": [
            {
                "Code": "time-span",
                "ShortText": "Time span",
                "LongText": "T i m e   s p a n",
                "Type": "number",
                "DefaultValue": "7",
                "Editor": {
                    "Type": "EnumEditor",
                    "Path": "@aulos/parameter-editors",
                    "Options": {
                        "comboValues": [
                            {
                                "label": "1 week",
                                "value": 7
                            },
                            {
                                "label": "2 weeks",
                                "value": 14
                            },
                            {
                                "label": "1 month",
                                "value": 28
                            },
                            {
                                "label": "2 months",
                                "value": 56
                            },
                            {
                                "label": "Forever",
                                "value": 9999
                            }
                        ]
                    }
                }
            }   
        ]
    }
    \end{lstlisting}
    \verb|filters| is an array of AULOS JSON \verb|Parameters| and \verb|metadata| which defines how the dto is modified for better frontend interpretation: \verb|fieldName| defines the name of the dto unique identifier while the properties field defines parameters that will be changed in the frontend such as Kendo chart field colors and field names.
    \item Define a \verb|DataSource| class to mediate between the controller and data access class/es [\ref{subsec:creazioneSource}].
    \item Extend the \verb|DiscoveryConfiguration| class of the desired type (extend base class if non-existant) [\ref{fig:exendDiscovery}]:
    \begin{itemize}
        \item Call \verb|AddSource| method in constructor:
        \begin{itemize}
            \item \verb|TDataSource| should be the type of your datasource.
            \item \verb|SourceCode| must be unique and the same as the previously created filename of the .json.
            \begin{lstlisting}[caption={TupleDiscoveryConfiguration example}, style=javaScriptCode]
public class TupleDiscoveryConfiguration : DiscoveryConfiguration
    {
        public TupleDiscoveryConfiguration()
        {
            AddSource(new SourceInfo
            {
                TDataSource = typeof(AllRefusedDataSource),
                SourceCode = "allcode",
                LongText = "All refused",
                ShortText = "All refused"
            });

            AddSource(new SourceInfo
            {
                TDataSource = typeof(RefusedTypeDataSource),
                SourceCode = "refusedtype",
                LongText = "Refused type",
                ShortText = "Refused type"
            });
        }
    }                
            \end{lstlisting}
        \end{itemize}
    \end{itemize}
    \item Define a controller if handling a new data type or otherwise use an already existant one [\ref{subsec:utilizzo}]:
    \begin{itemize}
        \item Single POST method taking \verb|GenericRequestDTO| argument.
        \item Call \verb|SourceManagementService| \verb|getDataSource<dataSource>| where \verb|dataSource| is the \verb|DataSource| class/interface you need.
        \begin{lstlisting}[caption={HttpPost request in Controller example}, style=javaScriptCode]
        [HttpPost]
        public IActionResult Search([FromBody] GenericRequestDTO request)
        {
            var dataSource = _sourceManagementService.GetDataSource<ITupleDataSource>(request.SourceCode);
            return new ObjectResult(dataSource.GetData(request.Filters));
        }
        \end{lstlisting}
    \end{itemize}
\end{itemize}
