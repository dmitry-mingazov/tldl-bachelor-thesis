\section{Topology}
\label{chap:topology}
I widget lato \textit{mobility} sono impiegati per mostrare i dati provenienti dalle macchine la cui struttura si articola in \textit{linee}, \textit{banchi} e \textit{stazioni}; queste ultime sono la parte del macchinario volta a effettuare i test sui vari componenti che attraversano la linea.

\subsection{ITopologyDataSource}
L'interfaccia \textit{ITopologyDataSource} contiene la definizione di tre metodi:

\begin{lstlisting}[caption={ITopologyDataSource.cs}, style=javaScriptCode]
    interface ITopologyDataSource
    {
        public List<int> GetLines();
        public List<int> GetBenches(int lineCode);
        public List<int> GetStations(int benchCode);
    }
\end{lstlisting}
Le classi che attualmente la implementano sono \textit{AllRefusedDataSource} e\\ \textit{RefusedTypeDataSource}, le quali andranno a richiamare i metodi necessari per la richiesta della lista dei codici di linee, banchi o stazioni, presente nella repository collegata a un database MySQL.

\subsection{Controller Topology}
Quando si procede con la creazione di un widget mobility è possibile scegliere la \textit{source}, in modo tale da definire la fonte in cui reperire i dati da visualizzare. In seguito,  facendo riferimento al formato standard dei macchinari, si ha la possibilità di scegliere il codice della linea, del banco e della stazione di cui si vogliono osservare i risultati.
Per poter scegliere questi codici vengono effettuate tre diverse chiamate al backend, in cui il controller definisce tre richieste \textit{HttpGet}:

\begin{lstlisting}[caption={TopologyController.cs}, style=javaScriptCode]
        [HttpGet("{sourceCode}/lines")]
        public IActionResult GetLines([FromRoute] string sourceCode)
        {
           var dataSource = _sourceManagementService
           .GetDataSource<ITopologyDataSource>(sourceCode);
            return new ObjectResult(dataSource.GetLines());
        }

        [HttpGet("{sourceCode}/benches/{lineCode}")]
        public IActionResult GetBenches([FromRoute] string sourceCode, int lineCode)
        {
            var dataSource = _sourceManagementService
            .GetDataSource<ITopologyDataSource>(sourceCode);
            return new ObjectResult(dataSource.GetBenches(lineCode));
        }

        [HttpGet("{sourceCode}/stations/{benchCode}")]
        public IActionResult GetStations([FromRoute] string sourceCode, int benchCode)
        {
            var dataSource = _sourceManagementService
            .GetDataSource<ITopologyDataSource>(sourceCode);
            return new ObjectResult(dataSource.GetStations(benchCode));
        }
\end{lstlisting}
Tramite il metodo \textit{GetDataSource<DataSource>} definito in \textit{SourceManagemetService}, viene creata l'istanza della \textit{DataSource} che implementa \textit{ITopologyDataSource}, in base al code passato come parametro.

Fornendo questa interfaccia come tipo, si ha la certezza che siano stati implementati i metodi volti al reperimento delle liste di codici di linee, banchi e stazioni.
\newpage