\section{Caching}
\label{ch:caching}
Il caso d'uso del progetto in questione prevede una moltitudine di dashboard collegate simultaneamente a un solo backend; al fine di  migliorare la performance è stato applicato un alleggerimento del carico tramite un \textit{middleware} \footnote{Il middleware è un software assemblato nella pipeline di app per gestire richieste e risposte. \cite{Middleware}} ASP.NET di cache. Ciò ha permesso la riduzione delle richieste effettivamente soddisfatte con dati generati appositamente da database SQL ed ElasticSearch sottostanti.
\subsection{Implementazione}
Il sistema di cache è stato implementato basandosi su un'implementazione già esistente di \verb|IMemoryCache| in ASP.NET Core. \cite{MemCache}
\begin{lstlisting}[caption={Startup.cs, Memory Cache Injection}, style=javaScriptCode]
services.AddMemoryCache(options => Configuration.Bind("Cache", options));
\end{lstlisting}

La chiave di cache viene generata dal frontend nell'interceptor http come \textit{hash} MD5 \cite{MD5} della concatenazione tra metodo \cite{HTTPMETHODS}, url con parametri \cite{HTTPPARAMS} e corpo \cite{HTTPBODY} della \textit{richiesta http} in uscita.
\begin{lstlisting}[caption={Http Interceptor, line 28}, style=javaScriptCode]
mergeMap(token => {
          this.md5.appendAsciiStr(request.method).appendAsciiStr(request.urlWithParams)
          .appendStr(JSON.stringify(request.body));
          request = request.clone({
            setHeaders: {
              Authorization: `Bearer ${token}`,
              'AULOS-hash': this.md5.end(false) as string
            }
          });
          return next.handle(request).pipe(catchError(error => {
            if (error instanceof HttpErrorResponse && error.status === 401) {
              return this.handle401Error(request, next);
            } else {
              return throwError(error);
            }
          
\end{lstlisting}

Il middleware di cache controlla se la richiesta in arrivo ha header \verb|AULOS-hash| e, se esistente, controlla se ha già una \textit{entry in cache} associata. In caso positivo, il middleware non chiama il middleware successivo nella catena e restituisce la entry salvata.
\begin{lstlisting}[caption={TotallyOriginalCachingMiddleware.cs, Cache hit scenario}, style=javaScriptCode]
context.Request.Headers.TryGetValue("AULOS-hash", out StringValues key);
            if (_cache.TryGetValue(key, out CustomCacheEntry entry))
            {
                await WriteResponse(context, entry);
                return;
            }
\end{lstlisting}

Per permettere allo sviluppatore backend di decidere a quali richieste limitare l'inserimento in cache, è stato implementato un attributo di cache estensione di \\ \verb|ActionFilterAttribute| \cite{ACTIONFILTER}, che, se anteposto a un controller o a un metodo http di un controller, permette di decorare il contesto della relativa richiesta http \cite{HTTPCONTEXT} in arrivo con flag \verb|ServerDuration| necessario a distinguere richieste abilitate all'inserimento in cache.
Il flag contiene la durata delle entry di cache relative all'azione a cui è applicato l'attributo di cache.

\begin{lstlisting}[caption={CacheAttribute.cs}, style=javaScriptCode, label={lst:cacheattribute}]

    public class CacheAttribute : ActionFilterAttribute
    {
        public int ServerDuration { get; set; }
        public override void OnActionExecuting(ActionExecutingContext context)
        {
            if (ServerDuration > 0)
            {
                context.HttpContext.Items["ServerDuration"] = $"{ServerDuration}";
            }
            else context.HttpContext.Items.Add("ServerDuration", 120);
            base.OnActionExecuting(context);
        }
    }
\end{lstlisting}

In caso negativo, il middleware lascia procedere la \textit{pipeline} e, se è presente nel contesto http \cite{HTTPCONTEXT} il flag \verb|ServerDuration| generato da attributi di cache \cite{ACTIONFILTER} (Fig.~\ref{lst:cacheattribute}), salva la risposta generata dalla pipeline come entry in cache di durata \verb|ServerDuration| e lascia procedere la pipeline. Se non sono presenti flag, il middleware lascia procedere la pipeline senza effettuare azioni.

\begin{lstlisting}[caption={TotallyOriginalCachingMiddleware.cs, Cache miss scenario}, style=javaScriptCode]
entry = await CaptureResponse(context);

       if (IsNotServerCachable(context))
           return;
       if (entry != null)
       {
           int serverCacheDuration = int.Parse((string)context.Items["ServerDuration"]);

           _cache.Set(key, entry, TimeSpan.FromSeconds(serverCacheDuration));
           }
       }
\end{lstlisting}
