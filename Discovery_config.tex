\section{Source Management}
\label{ch:sms}

Una \textit{source} è un'astrazione che rappresenta una sorgente di dati visualizzabili da un widget. Ogni source viene identificata da una DataSource, una classe che nel pattern \textit{Model View Controller} si colloca tra il Controller e il Model. La gestione delle source nel backend permette il riutilizzo di uno stesso widget per la visualizzazione di dati provenienti da contesti diversi.

\subsection{Creazione Source}
\label{subsec:creazioneSource}
Viene creata una classe DataSource che implementa l’interfaccia relativa al tipo di dato che deve ritornare. L'interfaccia presenterà un metodo volto a ottenere i dati richiesti (e.g. GetData), che verrà implementato dalla DataSource.
\begin{lstlisting}[caption={ITupleDataSource.cs}, style=sharpCode]
interface ITupleDataSource
    {
        public TupleDTO[] GetData(List<Parameter> parameters);
    }
\end{lstlisting}
\begin{lstlisting}[caption={AllRefusedDataSource.cs}, style=sharpCode]
public class AllRefusedDataSource : ITupleDataSource, ITopologyDataSource
{
  ...
  public TupleDTO[] GetData(List<Parameter> parameters)
  {
    var sam = new SeriousParameterHandler(parameters);
    var hours = sam.GetValue<long>("from");
    var now = DateTime.Now;
          DateTime from = now.AddHours(-hours);
    if (hours == -1)
          {
      from = new DateTime(0);
          }
    return sam.Contains("stationCode") ? 
      _repo.GetStationRefusedByCode(sam.GetValue<long>("stationCode"), from)
      .GetAwaiter().GetResult() :
      sam.Contains("benchCode") ? 
      _repo.GetBenchRefusedByCode(sam.GetValue<long>("benchCode"), from)
      .GetAwaiter().GetResult() :
      sam.Contains("lineCode") ?
      _repo.GetLineRefusedByCode(sam.GetValue<long>("lineCode"), from)
      .GetAwaiter().GetResult() :
              _repo.GetAllRefused(from)
      .GetAwaiter().GetResult();
  }
  ...
}
\end{lstlisting}

\subsection{Registrazione Source}
\label{subsec:source}
Le informazioni di una source sono descritte da SourceInfo e sono le seguenti:
\begin{itemize}
\item \textit{SourceCode} è l’identificativo della source;
\item \textit{TDataSource} è il tipo della DataSource legata alla source;
\item \textit{Parameters} sono i parametri messi a disposizione dalla source per modificare la \textit{query};
\item \textit{Metadata} sono i metadati volti a modificare come vengono visualizzati i dati lato frontend.
\end{itemize}
Parameters e Metadata vengono scritti in un file json all’interno della cartella \textit{properties} con il nome \textit{{SourceCode}.json}, come nel seguente esempio:
\begin{lstlisting}[caption={allcode.json}, style=JavaScriptCODE]
{
  "metadata": {
    "meta": {
      "fieldName": "label"
    },
    "properties": {
      "color": [
        {
          "field": "Passed",
          "value": "green"
        },
        {
          "field": "Refused",
          "value": "red"
        }
      ],
      ...
      ]
    }
  },
  "filters": [
    {
      "Code": "from",
      "ShortText": "In the last",
      "LongText": "From",
      "Type": "number",
      "DefaultValue": -1,
      "Editor": {
        "Type": "EnumEditor",
        "Path": "@aulos/parameter-editors",
        "Options": {
          "comboValues": [
            {
              "label": "12 hours",
              "value": 12
            },
            ...
            {
              "label": "Ever",
              "value": -1
            }
          ]
        }
      }
    }
  ]
} 
\end{lstlisting}
Per registrare una source, si crea una classe di configurazione che estende la classe astratta \textit{DiscoveryConfiguration} e si richiama all’interno del costruttore il metodo \textit{AddSource}, passandogli le informazioni della source che si sta registrando.
\\
\begin{lstlisting}[caption={TupleDiscoveryConfiguration.cs}, style=sharpCode]
public class TupleDiscoveryConfiguration : DiscoveryConfiguration
{
    public TupleDiscoveryConfiguration()
    {
        AddSource(new SourceInfo
        {
            TDataSource = typeof(AllRefusedDataSource),
            SourceCode = "allcode",
            LongText = "All refused",
            ShortText = "All refused"
        });

        AddSource(new SourceInfo
        {
            TDataSource = typeof(RefusedTypeDataSource),
            SourceCode = "refusedtype",
            LongText = "Refused type",
            ShortText = "Refused type"
        });
    }
}
\end{lstlisting}
Il metodo AddSource, definito in \textit{DiscoveryConfiguration}, ottiene attraverso \textit{PropertiesConfiguration} le eventuali informazioni aggiuntive contenute nell’apposito file \textit{{sourceCode}.json} per poi salvare la SourceInfo in una lista che verrà restituita attraverso il metodo pubblico Configure.
\begin{lstlisting}[caption={DiscoveryConfiguration.cs}, style=sharpCode]
public abstract class DiscoveryConfiguration
{
    private List<SourceInfo> _sources = new List<SourceInfo>();
    public List<SourceInfo> Configure()
    {
        return _sources;
    }

    protected void AddSource(SourceInfo source)
    {
        var sourceProperties = PropertiesConfiguration.GetFilters(source.SourceCode);
        source.Parameters = sourceProperties.Filters.ToList();
        source.Metadata = sourceProperties.Metadata;
        _sources.Add(source);
    }

}
\end{lstlisting}
DataDiscoveryConfiguration ottiene tutte le classi che estendono \textit{DiscoveryConfiguration}, per poi registrare in SourceManagementService tutte le source riscontrate.
\begin{lstlisting}[caption={DataDiscoveryConfiguraton.cs}, style=sharpCode]
public static class DataDiscoveryConfiguration
{
  public static void Configure(SourceManagementService sourceManagementService)
  {
      var configurations = AppDomain.CurrentDomain.GetAssemblies()
          .SelectMany(x => x.GetTypes())
          .Where(x => typeof(DiscoveryConfiguration).IsAssignableFrom(x) && 
                  !x.IsInterface && !x.IsAbstract)
          .ToList();

      configurations.ForEach(configuration =>
      {
          sourceManagementService
              .RegisterSources(
                  ((DiscoveryConfiguration)Activator.CreateInstance(configuration))
                  .Configure()
              );
      });

  }
}
\end{lstlisting}
\subsection{Interazioni Frontend}
\label{chap:frontend}
DataTypeCoupling è un enum che associa ad ogni risorsa il tipo di DataSource che utilizza.
\begin{lstlisting}[caption={DataTypeCoupling.cs}, style=sharpCode]
...

public enum DataTypeCoupling
{
    [CouplingData(Label = "tuple", DataType = typeof(ITupleDataSource))]
    tupleData,
    [CouplingData(Label = "orders", DataType = typeof(OrdersDataSource))]
    ordersData,

    ... 

}
\end{lstlisting}
DiscoveryController presenta due endpoint:
\begin{itemize}
\item \textit{{resource}/discovery} - restituisce SourceCode e le descrizioni delle source associate a una \textit{DataSource} del tipo associato a \textit{resource}, utilizzando il metodo \textit{GetDataSourcesByType} di \textit{SourceManagementService};
\item \textit{filters/{sourceCode}} - restituisce i parameters e i metadata di una specifica Source.
\end{itemize}
Il sourceCode restituito durante questa fase verrà utilizzato dal frontend per richiedere i dati.

\subsection{Utilizzo}
Un Controller che restituisce dati in una determinata forma deve chiamarsi \textit{{resource}Controller}, dove resource è una stringa registrata in DataTypeCoupling.
\begin{lstlisting}[caption={TupleController.cs}, style=sharpCode]
 [Route("api/widgets/[controller]")]
 [ApiController]
 public class TupleController: ControllerBase
  {
  
  ...
  
  }
\end{lstlisting}
Deve presentare un endpoint Post nella forma \textit{api/widgets/{resource}}
\begin{lstlisting}[caption={TupleController.cs, endpoint POST Search}, style=sharpCode]
[HttpPost]
public IActionResult Search([FromBody] GenericRequestDTO request)
{
    var dataSource = _sourceManagementService
      .GetDataSource<ITupleDataSource>(request.SourceCode);
    return new ObjectResult(dataSource.GetData(request.Filters));
}
\end{lstlisting}
Prende come Body una \textit{GenericRequest}, che presenta i seguenti campi:
\begin{itemize}
\item \textit{SourceCode} indica da quale fonte di dati devono essere presi i dati;
\item \textit{Filters} sono gli eventuali parametri necessari alla esecuzione della query.
\end{itemize}
\begin{lstlisting}[caption={GenericRequestDTO.cs}, style=sharpCode]
public class GenericRequestDTO
{
    public string SourceCode { get; set; }
    public List<Parameter> Filters { get; set; }
}
\end{lstlisting}
Restituisce la risorsa gestita dal Controller utilizzando il metodo \textit{GetData} della DataSource che ottiene attraverso il metodo \textit{GetDataSource<T>} di \textit{SourceManagementService}, passandogli il SourceCode contenuto nella richiesta.