\chapter{Implementazione}

\section{Qui}
Le informazioni di una source sono descritte da SourceInfo e sono le seguenti:
- SourceCode è l’identificativo della source
- TDataSource è il tipo della DataSource legata alla source
- Parameters sono i parametri messi a disposizione dalla source per modificare la query
- Metadata sono i metadati volti a modificare come vengono visualizzati i dati lato frontend

Parameters e Metadata vengono scritti in un file json all’interno della cartella {{DA_MODIFICARE}} con il nome {{SourceCode}}.json, come nel seguente esempio:

Per registrare una source, si crea una classe di configurazione che estenda la classe astratta DiscoveryConfiguration, e si richiama all’interno del costruttore il metodo AddSource, passandogli le informazioni della source che si sta registrando

Il metodo AddSource, definito in DiscoveryConfiguration, ottiene attraverso ParametersConfiguration le eventuali informazioni aggiuntive contenute nell’apposito file ‘{sourceCode}.json’ per poi salvare la SourceInfo in una lista che verrà restituita attraverso il metodo pubblico Configure

DataDiscoveryConfiguration ottiene tutte le classi che estendono DiscoveryConfiguration, per poi registrare in SourceManagementService tutte le source riscontrat
\newpage
Ciao Ciao


Ciao Ciao


\section{Pippo}




