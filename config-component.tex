\subsection{Widget Component}
Lo sviluppo di un Widget Component ha subito le seguenti modifiche architetturali:
\textbf{Logica acquisizione dati} precedentemente la logica con la quale un widget otteneva i dati veniva definita all'interno del Component, come in codice~\ref{lst:orders}. Attualmente invece viene 

\begin{lstlisting}[caption={Utilizzo metodo refreshData, orders.component.ts},label={lst:orders},style=javascriptCode]
export class OrdersComponent extends WidgetComponent<OrdersWidget> implements OnInit {

...

ngOnInit() {
        this.loading = true;
        this.updateRequest();
        this.widget.interestsParameterChanges.subscribe(() => {
            this.updateRequest();
            this.refreshData();
        });
        this.refreshData();
    }
    
    ...
    
protected refreshData(): void {
        this.ordersWidgetService.getOrders(this.request).subscribe(orders => {
            this.setOrdersView(this.checkUnseenOrders(orders));
            this.loading = false;
            this.changeDetectorRef.markForCheck();
        });
    }
\end{lstlisting}