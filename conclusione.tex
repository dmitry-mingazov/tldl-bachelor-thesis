\chapter{Conclusione}
\label{chap:conclusione}
In questo lavoro l'obiettivo principale è stato quello di creare una sovrastruttura al framework AULOS che permettesse agli sviluppatori di implementare in modo più efficiente e rapido le commesse nate dalle esigenze dei clienti Loccioni. Le problematiche affrontate sono state perlopiù relative alle scelte di design per trovare il giusto equilibrio tra la flessibilità del sistema e la facilità di configurazione. 
L'indagine del framework AULOS ha portato ad un incremento dei progressi del progetto, rendendo possibile il conseguimento del risultato finale. La documentazione Loccioni insieme all'assiduo sostegno dei collaboratori interni dell'impresa hanno fornito gli strumenti per decodificare il funzionamento del completo workflow del framework.
\\
La sovrastruttura creata permetterà di ridurre i tempi di sviluppo di una singola commessa e faciliterà il riuso di codice
scritto, minimizzando di conseguenza i costi per l'impresa. 
Il riuso sarà notevolmente favorito dalla possibilità di sviluppare widget che non presentino un forte accoppiamento verso l'origine dei dati che gestiscono.
\paragraph{Punti di miglioramento}
Il progetto sviluppato rappresenta una solida base per quanto concerne il lavoro di ridefinizione e generalizzazione dei widget. Per una futura estensione del risultati raggiunti, sono stati individuati i seguenti possibili sviluppi:
\begin{itemize}
    \item aggiunta di nuovi decorator all'interno dell'applicativo frontend per determinare ulteriori comportamenti comuni a più widget.
    \item successiva generalizzazione del backend con l'utilizzo di un singolo controller per la restituzione dei dati.
    \item definizione di uno "schema" e uso di un linguaggio più formale per i metadata e generale approfondimento della tematica.
    \item messa in atto di uno studio per testare varie strategie di caching durante un utilizzo reale della dashboard per capire come ottimizzarne maggiormente le performance.
\end{itemize}