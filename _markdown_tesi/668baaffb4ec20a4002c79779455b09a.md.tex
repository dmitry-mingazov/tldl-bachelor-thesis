\markdownRendererHeadingOne{:closed\markdownRendererUnderscore{}book: Multiuser Configurable DataView Web Dashboard - Backend}\markdownRendererInterblockSeparator
{}\markdownRendererHeadingTwo{Installation}\markdownRendererInterblockSeparator
{}\markdownRendererHeadingThree{Register Loccioni.Aulos packages repository}\markdownRendererInterblockSeparator
{}Just add on nuget sources:\markdownRendererInterblockSeparator
{}https://nuget.loccioni.com/repository/nuget-aulos-csharp-next/\markdownRendererInterblockSeparator
{}See: https://docs.microsoft.com/it-it/nuget/consume-packages/install-use-packages-visual-studio\markdownRendererHash{}package-sources\markdownRendererInterblockSeparator
{}\markdownRendererHeadingTwo{Widget Setup}\markdownRendererInterblockSeparator
{}\markdownRendererUlBegin
\markdownRendererUlItem Implement the data access layer\markdownRendererUlItemEnd 
\markdownRendererUlItem Add a \markdownRendererCodeSpan{\markdownRendererLeftBrace{}sourcecode\markdownRendererRightBrace{}.json} file in \markdownRendererCodeSpan{DiscoveryConfiguration/parameters}\markdownRendererUlItemEnd 
\markdownRendererUlItem Example: \markdownRendererCodeSpan{json
\markdownRendererLeftBrace{}
    "metadata": \markdownRendererLeftBrace{}
        "meta": \markdownRendererLeftBrace{}
            "fieldName": "label"
        \markdownRendererRightBrace{},
    "properties": \markdownRendererLeftBrace{}
        "color": [
            \markdownRendererLeftBrace{}
                "field": "Passed",
                "value": "green"
            \markdownRendererRightBrace{},
            \markdownRendererLeftBrace{}
                "field": "Refused",
                "value": "red"
            \markdownRendererRightBrace{}
        ],
        "field": [
            \markdownRendererLeftBrace{}
                "field": "Passed",
                "value": "ok"
            \markdownRendererRightBrace{},
            \markdownRendererLeftBrace{}
                "field": "Refused",
                "value": "ko"
            \markdownRendererRightBrace{}
        ]
    \markdownRendererRightBrace{}
    \markdownRendererRightBrace{},
    "filters": [
        \markdownRendererLeftBrace{}
            "Code": "time-span",
            "ShortText": "Time span",
            "LongText": "T i m e   s p a n",
            "Type": "number",
            "DefaultValue": "7",
            "Editor": \markdownRendererLeftBrace{}
                "Type": "EnumEditor",
                "Path": "@aulos/parameter-editors",
                "Options": \markdownRendererLeftBrace{}
                    "comboValues": [
                        \markdownRendererLeftBrace{}
                            "label": "1 week",
                            "value": 7
                        \markdownRendererRightBrace{},
                        \markdownRendererLeftBrace{}
                            "label": "2 weeks",
                            "value": 14
                        \markdownRendererRightBrace{},
                        \markdownRendererLeftBrace{}
                            "label": "1 month",
                            "value": 28
                        \markdownRendererRightBrace{},
                        \markdownRendererLeftBrace{}
                            "label": "2 months",
                            "value": 56
                        \markdownRendererRightBrace{},
                        \markdownRendererLeftBrace{}
                            "label": "Forever",
                            "value": 9999
                        \markdownRendererRightBrace{}
                    ]
                \markdownRendererRightBrace{}
            \markdownRendererRightBrace{}
        \markdownRendererRightBrace{}   
    ]
\markdownRendererRightBrace{}
} Where filters is an array of AULOS JSON parameters and metadata defines how the dto is modified for better frontend interpretation: \markdownRendererCodeSpan{fieldName} defines the name of the dto unique identifier while the properties field defines parameters that will be changed in the frontend such as Kendo chart field colors and field names.\markdownRendererUlItemEnd 
\markdownRendererUlItem Define a \markdownRendererCodeSpan{DataSource} class to mediate between the controller and data access class/es\markdownRendererUlItemEnd 
\markdownRendererUlItem Extend the \markdownRendererCodeSpan{DiscoveryConfiguration} class of the desired type (extend base class if non-existant)\markdownRendererUlItemEnd 
\markdownRendererUlItem Call \markdownRendererCodeSpan{AddSource} method in constructor\markdownRendererInterblockSeparator
{}\markdownRendererUlBeginTight
\markdownRendererUlItem \markdownRendererCodeSpan{TDataSource} should be the type of your datasource\markdownRendererUlItemEnd 
\markdownRendererUlItem \markdownRendererCodeSpan{SourceCode} must be unique and the same as the previously created filename of the .json ```csharp public class TupleDiscoveryConfiguration : DiscoveryConfiguration \markdownRendererLeftBrace{} public TupleDiscoveryConfiguration() \markdownRendererLeftBrace{} AddSource(new SourceInfo \markdownRendererLeftBrace{} TDataSource = typeof(AllRefusedDataSource), SourceCode = "allcode", LongText = "All refused", ShortText = "All refused" \markdownRendererRightBrace{});\markdownRendererUlItemEnd 
\markdownRendererUlEndTight \markdownRendererUlItemEnd 
\markdownRendererUlEnd \markdownRendererInterblockSeparator
{}\markdownRendererInputVerbatim{./_markdown_tesi/9fe3acca3b92d148cda52fa662c309a2.verbatim}\markdownRendererInterblockSeparator
{}\markdownRendererUlBeginTight
\markdownRendererUlItem Define a controller if handling a new data type or otherwise use an already existant one \markdownRendererUlItemEnd 
\markdownRendererUlItem Single POST method taking \markdownRendererCodeSpan{GenericRequestDTO} argument\markdownRendererUlItemEnd 
\markdownRendererUlItem Call \markdownRendererCodeSpan{SourceManagementService} \markdownRendererCodeSpan{getDataSource<dataSource>} where \markdownRendererCodeSpan{dataSource} is the \markdownRendererCodeSpan{DataSource} class/interface you need \markdownRendererCodeSpan{csharp
[HttpPost]
    public IActionResult Search([FromBody] GenericRequestDTO request)
    \markdownRendererLeftBrace{}
        var dataSource = \markdownRendererUnderscore{}sourceManagementService.GetDataSource<ITupleDataSource>(request.SourceCode);
        return new ObjectResult(dataSource.GetData(request.Filters));
    \markdownRendererRightBrace{}
}\markdownRendererUlItemEnd 
\markdownRendererUlEndTight \relax