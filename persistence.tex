\section{Persistence}
\label{chap:persistence}
Il Persistence Layer, in letteratura \textit{Data Access Layer}, è utilizzato per la manipolazione dei dati archiviati in una memoria persistente.\\
La corrispondenza tra modello object oriented (\textit{domain model}) e schema del database è data in gestione ad un framework \textit{ORM} (Object-Relational Mapping) \cite{EF}: Entity Framework, il quale ha la funzione di semplificare le operazioni di interrogazione e manipolazione dei dati.

\subsection{Struttura Persistence}
\begin{figure}[h!]
\begin{center}
  \includegraphics[width=8cm]{images/Persistence.jpg}\\
  \caption{Formato cartella persistence}\label{fig:persistence}
\end{center}
\end{figure}

Nella conformazione che si denota in figura~\ref{fig:persistence}, ciascuna cartella definisce un contesto applicativo ed è articolata in:

\begin{itemize}
\item 
\textbf{Entity configurations:} vi sono le configurazioni delle entità presenti in \textbf{Models} e vi si effettua il \textit{mapping} di tabelle e colonne nel database.
\item
\textbf{Models:} ciascun modello è un'entità che rappresenta l'analogo del record, o riga, di una tabella.
\item 
\textbf{Repositories:} volte all'interrogazione e manipolazione dati tramite l'uso di \textit{Data Manipulation Language} (quali Linq e Nest, rispettivamente per database SQL ed ElasticSearch).
\item 
\textbf{IUnitOfWork:} interfaccia in cui vengono definite \textit{repository} e metodo \verb|SaveChanges|.
\item
\textbf{DbContext:} in questa classe viene fornita la \textit{connection string} del database (definita in \verb|appsettings|), si dichiarano i DbSet relativi ai model, che sono entity set utilizzati per le \textit{CRUD operation} \cite{DbSet} e in \verb|OnModelCreating| si applicano le configurazioni dei model.
\item 
\textbf{UnitOfWork:} implementa l'interfaccia creando la repository e applicando il metodo \verb|SaveChanges| al context; quest'ultimo ha la funzione di salvare un insieme di modifiche in una transazione, oppure di eseguire un \textbf{rollback} (ossia un ripristino allo stato che precede l'aggiornamento).
\end{itemize}
\verb|PersistenceServiceCollectionExtension| si occupa di configurare le \textit{dependency injection} per la creazione del context ElasticSearch.