\FloatBarrier
\section{Elasticsearch}
\begin{figure}[h!]
\centering
\includegraphics[scale=0.05]{images/esearch.png}
\caption{Logo Elasticsearch}
\end{figure}
Elasticsearch è un motore di ricerca e analisi full-text open source altamente scalabile. Consente di archiviare, cercare e analizzare grandi volumi di dati rapidamente e quasi in tempo reale. Viene generalmente utilizzato come motore sottostante per applicazioni che hanno caratteristiche e requisiti di ricerca complessi.
Elasticsearch dispone di diversi casi d'uso \cite{ELASTICSEARCH}:
\begin{itemize}
    \item consente di avere una visione più ampia delle informazioni a disposizione senza risultare dispersivo, anche con agglomerati di dati estremamente consistenti grazie alle "aggregazioni".
    \item combina diversi tipi di ricerche: strutturate, non strutturate, geografiche, ricerca di applicazioni, analisi della sicurezza, metriche e registrazione.
    \item si adatta a qualsiasi situazione che va dal semplice computer con un singolo nodo fino ad arrivare ad un cluster con centinaia di server, rendendo la creazione di un prototipo più agevole.
    \item utilizza API RESTful standard e JSON. L'apporto costante della comunità ha portato alla creazione e al mantenimento di client in molti linguaggi come Java, Python, .NET, SQL, Perl, PHP.
    \item è possibile utilizzare le funzionalità di ricerca e analisi in tempo reale di Elasticsearch per lavorare sui big data utilizzando il connettore Elasticsearch-Hadoop (ES-Hadoop).
\end{itemize}
Data l'elevata efficienza di questo software diverse compagnie informatiche l'hanno scelto per i loro prodotti. Gli ambiti interessati sono quelli di ricerca per quanto concerne Tinder, dove Elasticsearch permette di rendere il sistema più reattivo e veloce, mentre nel caso di Netflix l'utilizzo principale è focalizzato al giusto indirizzamento di notifiche push, email e messaggistica \cite{ELASTICSEARCH}.\\
Nel progetto discusso, Elasticsearch è stato utilizzato nell'implementazione del layer di accesso dei dati relativi alle macchine che testano le componenti degli elettrodomestici Whirlpool. In sinergia con questo strumento è stata impiegata la dashboard di visualizzazione dati Kibana.
\FloatBarrier
