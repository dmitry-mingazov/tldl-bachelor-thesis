\FloatBarrier
\pagebreak
\section{Configurazione}
Attualmente la nuova architettura sviluppata offre la possibilità di creare e configurare un widget in modo più strutturato, scrivendo molte meno linee di codice.
In precedenza il processo per implementare un widget era prettamente verticale e si articolava nei seguenti step:
\begin{itemize}
\item
Creazione component widget
\begin{itemize}
\item
Parte grafica HTML.
\item
Gestione interfacciamento con il widget.
\item
Interfacciamento con il service.
\item
Intera gestione della logica di acquisizione dati.
\end{itemize}
\item
Creazione widget
\begin{itemize}
\item
Creazione e registrazione di parametri.
\item
Messa a disposizione di valori dei parametri e relativi observable.
\end{itemize}
\item
Creazione module
\begin{itemize}
\item
Registrazione widget.
\end{itemize}
\item
Creazione service
\begin{itemize}
\item
Implementazione logica di acquisizione dati (richieste HTTP).
\end{itemize}
\end{itemize}
Il backend non presentava una struttura definita per l'accesso dei dati. Rimaneva a carico dello sviluppatore implementarne la logica e scrivere un Controller attraverso il quale il frontend avrebbe potuto ottenere i dati raccolti.\\\\
Allo stato attuale al fine di sviluppare un widget un programmatore deve seguire i seguenti passi:
\begin{itemize}
    \item Frontend
\begin{itemize}
    \item Creazione component widget estendendo la classe BaseWidgetComponent
    \begin{itemize}
        \item BaseWidgetComponent aggiunge la logica di inizializzazione dei dati da far visualizzare.
    \end{itemize}
    \item Creazione widget estendendo la classe BaseWidget
    \begin{itemize}
        \item BaseWidget aggiunge la creazione e l'inizializzazione dei parametri legati all'ottenimento dei dati.
        \item decorazione della classe con il decorator factory WidgetDecorator.
    \end{itemize}
    \item Creazione module
    \begin{itemize}
        \item Registrazione widget.
    \end{itemize}
    \item Creazione adapter (opzionale)
\end{itemize}
\item Backend
\begin{itemize}
	\item Implementare il data access layer
	\item Aggiungere il \verb|sourcecode.json| relativo in \verb|DiscoveryConfiguration/parameters|
	\item Definire una classe \verb|DataSource|, mediatrice tra controller e data access
	\item Estendere \verb|DiscoveryConfiguration| con la classe del tipo desiderato
	\item Definire un nuovo controller per manipolare un nuovo data type
\end{itemize}
\end{itemize}