\FloatBarrier
\pagebreak
\section{Configurazione}
Precedentemente, lo sviluppo -- prettamente verticale -- di un widget e la procedura di accesso dati per popolarlo di dati si articolava in:

\begin{multicols}{2}
Prima:
\begin{itemize}
\item
Creazione component widget
\begin{itemize}
\item
Parte grafica HTML
\item
Gestione interfacciamento con il widget
\item
Interfacciamento con il service
\item
Gestire interamente la logica di acquisizione dati
\end{itemize}
\item
Creazione widget
\begin{itemize}
\item
Creazione e registrazione di parametri
\item
Messa a disposizione di valori dei parametri e relativi observable
\end{itemize}
\item
Creazione module
\begin{itemize}
\item
Registrazione widget
\end{itemize}
\item
Creazione service
\begin{itemize}
\item
Implementazione logica di acquisizione dati (Richieste http et cetera)
\end{itemize}
\item
Registrazione widget su module
\end{itemize}
\columnbreak

Dopo:
\begin{itemize}
\item
Creazione component widget
\begin{itemize}
\item
Parte grafica HTML
\item
Definizione callback di aggiornamento dati
\end{itemize}
\item
Creazione widget
\item
Creazione module
\item
Creazione service
\item
Registrazione widget su module
\end{itemize}

\end{multicols}

\subsection{Widget Component}
Lo sviluppo di un Widget Component ha subito le seguenti modifiche architetturali:
\textbf{Logica acquisizione dati} precedentemente la logica con la quale un widget otteneva i dati veniva definita all'interno del Component, come in codice~\ref{lst:orders}. Attualmente invece viene 

\begin{lstlisting}[caption={Utilizzo metodo refreshData, orders.component.ts},label={lst:orders},style=javascriptCode]
export class OrdersComponent extends WidgetComponent<OrdersWidget> implements OnInit {

...

ngOnInit() {
        this.loading = true;
        this.updateRequest();
        this.widget.interestsParameterChanges.subscribe(() => {
            this.updateRequest();
            this.refreshData();
        });
        this.refreshData();
    }
    
    ...
    
protected refreshData(): void {
        this.ordersWidgetService.getOrders(this.request).subscribe(orders => {
            this.setOrdersView(this.checkUnseenOrders(orders));
            this.loading = false;
            this.changeDetectorRef.markForCheck();
        });
    }
\end{lstlisting}
\subsection{Backend}

