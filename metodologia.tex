\chapter{Metodologia di sviluppo}
\label{ch:metodologia}
Per lo sviluppo del progetto in gruppo è stata ritenuta necessaria una pianificazione solida del lavoro che non portasse a perdite di efficienza, in quanto ciò avrebbe comportato ritardi nella consegna di feature concordate con clienti interni all'impresa.
Per questo si è deciso di utilizzare un'organizzazione simile a quella \textit{Scrum}, senza però la creazione di artefatti.

Il lavoro è stato caratterizzato da discussioni collettive giornaliere sui \textit{task} da portare a termine, report giornalieri all'interno del team riguardanti i moduli software realizzati, e feedback costante e capillare per quanto riguarda la pertinenza del lavoro svolto alle loro richieste.

Il lavoro è stato quindi organizzato in maniera assimilabile alla metodologia \textit{Agile} \textit{Scrumban}.
\section{Scrumban}
Scrumban nasce come ibrido tra gli approcci allo sviluppo di \textit{Scrum} e \textit{Kanban}

\begin{multicols}{2}
Scrum:\\
\begin{itemize}
\item
Dividere l'organico in piccoli team multidisciplinari in grado di stabilire un'organizzazione interna autonomamente
\item
Dividere il lavoro in una lista di piccoli moduli consegnabili ordinati per priorità, stimando l'impegno richiesto per ciascuno
\item
Dividere il tempo in brevi \textit{sprint}, che potenzialmente si concludono con codice pronto ad andare in produzione
\item
Basandosi sull'analisi del prodotto dell'iterazione, ottimizzare il piano di produzione e aggiornare le priorità comunicando con il cliente
\item
Ottimizzare il processo attraverso una analisi retrospettiva dopo ogni iterazione
\end{itemize}
\columnbreak
Kanban:
\begin{itemize}
\item
Visualizzare il \textit{workflow}
\begin{itemize}
\item
Dividere il lavoro in pezzi, scrivere ciascuno su di un supporto fisico ed esporli visivamente
\item
Usare colonne con nomi significativi in modo da distinguere la fase di sviluppo di ogni pezzo
\end{itemize}
\item
Limitare il lavoro in corso assegnando limiti a quante task possono essere svolte in simultanea
\item
Misurare il tempo di svolgimento di ciascuna task in modo da poter minimizzare e rendere il più prevedibile possibile i tempi di svolgimento di task future
\end{itemize}
\end{multicols}

Scrumban, come combinazione di Scrum e Kanban, porta ad avere parte dei vantaggi di entrambi, quali la natura prescrittiva di Scrum e i meccanismi di ottimizzazione di processo di Kanban.

I diagrammi di flusso e le rappresentazioni dei processi mostrano le debolezze e offrono opportunità di migliorare i singoli processori.
Se il tempo medio delle singole iterazioni viene tenuto sotto controllo e le capacità del team sono bilanciate rispetto all'obiettivo, lo sviluppo complessivo del progetto rispetterà le scadenze prefissate.
Il team utilizza una coda di lavori pronti a essere iniziati, in modo tale che in ogni momento ci sia sempre qualcosa di significativo da svolgere. \cite{SCRUMBAN}
\paragraph{Vantaggi}
\begin{itemize}
\item Qualità
\item \textit{Just-in-time}: ogni cosa viene decisa e svolta solo quando necessario
\item Riduzione del tempo complessivo di sviluppo
\item \textit{Kaizen}: miglioramento continuo
\item Riduzione delle attività che non portano valore al cliente
\item Miglioramento dei processi aggiungendo valore da Scrum quando necessario
\end{itemize}
In conclusione la scelta dello Scrumban si è rivelata ideale grazie al valore che i suoi vantaggi portano in progetti caratterizzati da forti elementi di ricerca e sviluppo, come quello esposto in questo elaborato. 