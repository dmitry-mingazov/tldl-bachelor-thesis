\chapter{Metodologia di sviluppo}
\label{ch:metodologia}
Per lo sviluppo del progetto in gruppo è stata ritenuta necessaria una pianificazione solida del lavoro che non portasse a perdite di efficienza, in quanto ciò avrebbe comportato ritardi nella consegna di feature concordate con clienti interni all'impresa.
Per questo si è deciso di utilizzare un'organizzazione simile a quella Scrum, senza pero' la creazione di artefatti.

[WIP]Il lavoro e' stato caratterizzato da discussioni collettive giornaliere [WIP]sui task da portare a termine, report giornalieri all'interno del team [WIP]riguardanti i moduli software realizzati, e feedback costante e [WIP]capillare per quanto riguarda la pertinenza del lavoro svolto alle loro [WIP]richieste.

Il lavoro è stato organizzato in maniera assimilabile alla metodologia Agile Scrumban
\section{Scrumban}
Scrumban nasce come ibrido tra gli approcci allo sviluppo di Scrum e Kanban

[WIP]
\begin{multicols}{2}
Scrum:
\begin{itemize}
\item
Dividere l'organico in piccoli team multidisciplinari in grado di stabilire un'organizzazione interna autonomamente
\item
Dividere il lavoro in una lista di piccoli moduli consegnabili ordinati per priorità, stimando l'impegno richiesto per ciascuno
\item
Dividere il tempo in brevi iterazioni di durata fissa, potenzialmente con codice pronto ad andare in produzione
\item
Basandosi sul prodotto dell'analisi della release dopo ogni iterazione, ottimizzare il piano di release e aggiornare le priorità collaborando con il cliente
\item
Ottimizzare il processo analizzando le iterazioni una volta completate
\end{itemize}
\columnbreak
Kanban:
\begin{itemize}
\item
Visualizzare il workflow
\begin{itemize}
\item
Dividere il lavoro in pezzi, scrivere ciascuno su di un supporto fisico ed esporli visivamente
\item
Usare colonne con nomi significativi in modo da distinguere la fase di sviluppo di ogni pezzo
\end{itemize}
\item
Limitare il lavoro in corso, assegnare limiti a quanti task in simultanea possono essere portati avanti
\item
Misurare il tempo di svolgimento di ciascun task in modo da poter minimizzare e rendere il più prevedibile possibile i tempi di svolgimento di task futuri
\end{itemize}
\end{multicols}

Scrumban, come combinazione di Scrum e Kanban, porta ad avere parte dei vantaggi di entrambi, quali la natura prescrittiva di Scrum e i meccanismi di ottimizzazione di processo di Kanban.
I diagrammi di flusso e le rappresentazioni dei processi mostrano le debolezze e offrono opportunità di migliorare i singoli processori.
Se il tempo medio delle singole iterazioni viene tenuto sotto controllo, e le capacità del team sono bilanciate rispetto all'obiettivo, lo sviluppo complessivo del progetto rispetterà le scadenze prefissate.
Considerando che il team utilizza una coda di lavori pronti a essere iniziati, in ogni momento ci sarà sempre qualcosa che valga la pena fare.
\\
