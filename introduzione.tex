\chapter{Introduzione}
\label{chap:intro}

In questa relazione verrà esposto il lavoro svolto durante il periodo di Stage++ in Loccioni.
La principale attività aziendale della Loccioni è quella di progettare e realizzare sistemi di misura per 
testare e controllare componenti di auto, lavatrici ed altri strumenti elettronici.
Queste attività sono svolte in "linee di test"; ogni "linea" è composta da più "banchi" che a loro volta sono composti da più "stazioni", all'interno delle quali avviene 
il test effettivo del pezzo, o del DuT (device under test).\\
I risultati dei test sono poi visualizzabili tramite widget all'interno di dashboard remote, prendendo i dati da un server centrale, 
o collocate in prossimità della macchina.


\section{Obiettivi}
L'obiettivo del progetto è stato quello di realizzare, partendo dal framework AULOS Loccioni, un ambiente di editing per
Dashboard Web con i relativi widget per la visualizzazione dei dati provenienti dalle macchine Loccioni e dalla gestione 
interna dei flussi gestionali di materiali e di controllo. Ciò ha comportato inoltre il dover cercare e pensare al giusto equilibrio 
tra flessibilità del sistema, per rendere più veloce ed immediato lo sviluppo di widget, e facilità di configurazione.

\section{Strumenti e tecnologie}
    \begin{multicols}{3}
    \begin{itemize}
        \item Git
        \item Gitlab
        \item Visual Studio
        \item Visual Studio Code
        \item Postman
        \item JMeter
        \item HeidiSQL
        \item MS SQL SMS
        \item Kibana
        \item ASP.NET Core 3.1
        \item Angular 10
        \item NodeJs
        \item Entity Framework
        \item ElasticSearch
        \item AULOS
        \item Nest
        \item Newtonsoft.Json
        \item KendoUI
    \end{itemize}
    \end{multicols}

