\section{Security}
\label{chap:security}

\paragraph{Module}
L'autenticazione degli utenti all'interno della dashboard è gestita dall' \verb|ItSecurity|-\verb|Module|. Nell'\verb|ItSecurityModule| vengono registrati nell'array \verb|providers|, che gestisce la dependency injection, i service necessari specificando per ognuno di essi la classe astratta a cui fanno riferimento e la classe d'implementazione.
\begin{lstlisting}[caption={Injection dei service nell'ItSecurityModule}, style=javaScriptCode]
@NgModule({
  imports: [HttpClientModule],
})
export class ItSecurityModule {
  /**
   * Main configuration to call on root application module
   */
  public static forRoot(): ModuleWithProviders<ItSecurityModule> {
    return {
      ngModule: ItSecurityModule,
      providers: [
        {
          provide: AbstractUserFactory,
          useClass: ItUserFactory
        },
        {
          provide: AbstractTokenProviderService,
          useClass: ItAccessTokenProviderService,
        },
        {
          provide: ItRefreshTokenProviderService,
          useClass: ItRefreshTokenProviderService,
        },
        {
          provide: AbstractSecurityService,
          useClass: ItSecurityService
        },
        {
          provide: HTTP_INTERCEPTORS,
          useClass: ITHttpSecurityInterceptor,
          multi: true
        },
        {
          provide: APP_INITIALIZER,
          useFactory: ItSecurityModule.appInitializer,
          deps: [AbstractTokenProviderService, ItRefreshTokenProviderService, 
            AbstractSecurityService],
          multi: true
        }
      ]
    };
  }
  ...
}
\end{lstlisting}
In \verb|providers| vengono fornite le implementazioni delle classi astratte
\verb|Abstract|-\verb|SecurityService|, \verb|AbstractUserFactory| e \verb|AbstractTokenProviderServi|-\verb|ce|, parte del framework Aulos.
Tutta la logica principale dell'autenticazione risiede all'interno dell'\verb|ItSecurityService|. La sua responsabilità è quella di autenticare e disconnettere l'utente nella \textit{dashboard}. \\
\begin{lstlisting}[caption={Principali metodi della classe ItSecurityService}, style=javaScriptCode]
@Injectable({
  providedIn: 'root'
})
export class ItSecurityService extends AbstractSecurityService {

  ...

  login(username: string, password: string): Observable<ItUser> {
  
    ...
    
    return this.httpClient.post<JWTToken>(`${this.serviceConfiguration.baseUrl}/
    issue/oauth2/token`, { username, password, grant_type: 'password', 
        scope: 'urn:gl-services-infrastructure' }, { headers })
      .pipe(
        map(response => {
          const jwtUser = this.decodeToken(response.access_token);
          /* We have to set autologin to false to be sure to enable login phase 
          (when default user logout)*/
          this.serviceConfiguration.autoLogin = false;
          // Store current token so next call will have correct token
          this.accessTokenProviderService.store(btoa(JSON.stringify(jwtUser)));
          this.refreshTokenProviderService.store(response.refresh_token);
          this.currentJWTUser.next(jwtUser);
          return jwtUser;
        }),
        concatMap(jwtUser => {
          return this.getUser(jwtUser.loginName)
            .pipe(map(user => {
              this.currentUser.next(user);
              return user;
            }));
        }));
  }

  logout(): Observable<unknown> {
    this.accessTokenProviderService.remove();
    this.refreshTokenProviderService.remove();
    this.currentUser.next(null);
    return EMPTY;
  }

}
\end{lstlisting}
\paragraph{Grant}
Si è utilizzato il \textit{Resource Owner Password Credentials Grant} \cite{GRANT} in quanto nel caso d'uso in questione si ha un elevato grado di fiducia tra frontend e backend.
Le credenziali vengono mandate all'\textit{Authorization Server} che restituisce, se le credenziali sono corrette, un \textit{access token} e un \textit{refresh token}.\\
\begin{figure}[h]
\begin{center}
\label{fig:grantflow}
\begin{verbatim}
+----------+
| Resource |
|  Owner   |
|          |
+----------+
     v
     |    Resource Owner
    (A) Password Credentials
     |
     v
+---------+                                  +---------------+
|         |>--(B)---- Resource Owner ------->|               |
|         |         Password Credentials     | Authorization |
| Client  |                                  |     Server    |
|         |<--(C)---- Access Token ---------<|               |
|         |    (w/ Optional Refresh Token)   |               |
+---------+                                  +---------------+

\end{verbatim}
\caption{Resource Owner Password Credentials Flow \cite{GRANT}}
\end{center}
\end{figure}
\FloatBarrier
\begin{figure}[h]
\begin{center}
\label{fig:refreshflow}
\begin{verbatim}
+--------+                                   +---------------+
|        |--(A)---- Authorization Grant ---->|               |
|        |                                   |               |
|        |<-(B)--------- Access Token -------|               |
|        |             & Refresh Token       |               |
|        |                                   |               |
|        |                      +----------+ |               |
|        |--(C)- Access Token ->|          | |               |
|        |                      |          | |               |
|        |<(D)- Protected Reso -| Resource | | Authorization |
| Client |                      |  Server  | |     Server    |
|        |--(E)- Access Token ->|          | |               |
|        |                      |          | |               |
|        |<(F)Invalid Token Err-|          | |               |
|        |                      +----------+ |               |
|        |                                   |               |
|        |--(G)----- Refresh Token --------->|               |
|        |                                   |               |
|        |<-(H)------ Access Token ----------|               |
+--------+      & Optional Refresh Token     +---------------+

\end{verbatim}
\caption{Token Refresh Flow \cite{REFRESH}}
\end{center}
\end{figure}
\FloatBarrier 
Il refresh token viene utilizzato dal Client per richiedere all'Authorization Server un nuovo access token quando quest'ultimo è scaduto.
La memorizzazione dell'access e del refresh token viene svolta rispettivamente dall'\verb|ItAccessTokenProviderService| e dall'\verb|ItRefreshTokenProviderService|.
Entrambe le classi estendono la classe astratta \verb|AbstractTokenProviderService|.\\
\begin{lstlisting}[caption={Classe astratta AbstractTokenProviderService}, style=javaScriptCode]
export declare abstract class AbstractTokenProviderService {
    /**
     * Get token from storage
     */
    abstract get(): Observable<string>;
    /**
     * Store token on storage
     */
    abstract store(token: string): Observable<void>;
    /**
     * Remove token on storage
     */
    abstract remove(): Observable<void>;
}
\end{lstlisting}
All'avvio della dashboard l'\verb|ItSecurityModule| autentica automaticamente l'utente se l'access token è già presente nel \textit{local storage}.\newline
\begin{lstlisting}[caption={Login con token nell'ItSecurityModule}, style=javaScriptCode]
@NgModule({
  imports: [HttpClientModule],
})
export class ItSecurityModule {
    
  ...

  public static appInitializer(
    accessTokenProviderService: AbstractTokenProviderService,
    refreshTokenProviderService: ItRefreshTokenProviderService,
    securityService: AbstractSecurityService
  ): () => Promise<void> {
    const result = (): Promise<void> => {
      return new Promise((resolve, reject) => {
        // Check if token is already present on browser storage
        const sources = [accessTokenProviderService.get(), 
            refreshTokenProviderService.get()];
        forkJoin(sources).subscribe((val) => {
          const accTok: string = val[0];
          const refrTok: string = val[1];
          if (accTok && refrTok) {
            securityService.loginWithToken((JSON.parse(atob(accTok)) as ItJWTUser)
                .accessToken).subscribe(() => resolve());
          } else {
            resolve();
          }
        }, (error) => console.error(error));
      });
    };
    return result;
  }
}
\end{lstlisting}
\pagebreak
\paragraph{Interceptor} Chi intercetta e gestisce le \textit{richieste} e le \textit{risposte http} è l'\verb|ItHttpSecurityIntercep|-\verb|tor|.
Nello specifico l'\verb|ItHttpSecurityInterceptor| si occupa di: 
\begin{itemize}
    \item decorare le richieste http in uscita con token di autorizzazione, se l'header di autenticazione non è già presente nella richiesta,
    \item richiedere un access token tramite la procedura di refresh token in caso di risposta 401 a una richiesta al backend,
    \item ripetere la richiesta fallita con codice 401 dopo aver ottenuto il nuovo access token, 
    \item effettuare il logout in caso di fallimento della procedura di refresh.
\end{itemize}