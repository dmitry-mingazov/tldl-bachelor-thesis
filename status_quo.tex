\chapter{Architettura legacy [WIP]}
%In questo capitolo verrà discussa la struttura precedentemente utilizzata %nell'applicativo e [...]
%L'applicativo presentava una parte backend che aveva la responsabilità di 
%fornire degli endpoint REST attraverso i quali l'applicativo frontend potesse %ottenere i dati da far visualizzare all'utente finale attraverso i widget
\section{Struttura del widget}
Un widget è una piccola applicazione, semplice e immediata, che mostra all'utente dati e informazioni, di solito sotto forma di grafici o tabelle.
La sua struttura si articolava in diversi file con scopi diversi:
\begin{itemize}
\item \textbf{widget.component.html} contiene il codice HTML che descrive l'aspetto grafico che il widget avrà all'interno della dashboard, e rappresenta i dati contenuti dal widget.component.ts
\item \textbf{widget.component.ts} è un Component Angular, contiene i dati che il widget deve visualizzare e si occupa di ottenere i dati interfacciandosi con il servizio opportuno
\item \textbf{widget.ts} è la classe che gestisce la configurazione del widget
\item \textbf{widget.service.ts} è un servizio Angular, contiene la logica di accesso ai dati che il widget deve visualizzare
\item \textbf{widget.module.ts} è un modulo Angular, contiene la dichiarazione del componente widget e la sua registazione al WidgetService
\end{itemize}

\subsection{Widget Component}
Il Component di un widget è una classe che deve estendere WidgetComponent del framework Aulos e avere come tipo generico una classe che estenda Widget. Nel codice~\ref{lst:booking_view_component} si può osservare \verb|BookingViewComponent| che prende come tipo generico \verb|BookingViewWidget|
\begin{lstlisting}[caption={orders.component.ts},label={lst:booking_view_component},style=javascriptCode]
@Component({
    selector: 'app-booking-view-widget',
    templateUrl: './booking-view.component.html',
    encapsulation: ViewEncapsulation.None,
    changeDetection: ChangeDetectionStrategy.OnPush
})
export class BookingViewComponent extends WidgetComponent<BookingViewWidget> implements OnInit {

	...

}
\end{lstlisting}
\subsection{Widget}
Il Widget è una classe che deve estendere Widget del framework Aulos, avendo così modo di interfacciarsi con la dashboard Aulos. La sua responsabilità è quella di creare e gestire i parametri relativi al widget (~\ref{chap:parameters}).
\begin{lstlisting}[caption={booking-view-widget.ts},style=javascriptCode]
export class BookingViewWidget extends Widget {

    private timeRangeParameter: NumberParameter;

    constructor(injector: Injector, widgetDescriptor: WidgetDescriptor) {
        super(injector, widgetDescriptor);
        this.setSize(500, 150);
    }

    public get timeRange(): number {
        return this.timeRangeParameter.value;
    }

    public get timeRangeObservable(): Observable<number> {
        return this.timeRangeParameter.valueChanges;
    }

    protected createConfigParameters(): Observable<Parameter<unknown>[]> {
        this.createFilterDateParameter();
        return super.createConfigParameters();
    }

    private createFilterDateParameter() {
        this.timeRangeParameter = new NumberParameter('timeRange', 'Time Range', 15);
        this.timeRangeParameter.descriptor.editor = new EditorDescriptor(
            EDITOR_BASE_PATH,
            'EnumEditor',
            this.generateEnumOptions()
        );
        this.configParameters.push(this.timeRangeParameter);
    }

    private generateEnumOptions(): EnumOptions {
        return {
            comboValues: [
                { label: "1 week", value: 7 },
                { label: "2 weeks", value: 15 },
                { label: "1 month", value: 30 },
                { label: "2 months", value: 60 },
                { label: "Forever", value: 9999 },
            ]
        }
    }

}
\end{lstlisting}

\subsection{Widget Service}
Il widget Service è un servizio che gestisce l'interazione con il backend al fine di ottenere i dati che il widget deve visualizzare. Nel codice~\ref{lst:booking_view_service} si può osservare il servizio che ottiene i dati per il widget BookingView.
\begin{lstlisting}[caption={booking-view-widget.service.ts},label={lst:booking_view_service},style=javascriptCode]
@Injectable({
  providedIn: 'root'
})
export class BookingViewWidgetService {

  private baseUrl: string = environment.apiUrl;
  private widgetsUrl = `${this.baseUrl}/${environment.widgetsUrl}`;

  constructor(
    private http: HttpClient,
    private adapter: BookingInfoAdapter
  ) {
  }

  public getBookingViewInfo(id: number): Observable<BookingInfo[]> {
    return this.http.get<BookingDTO[]>(`${this.widgetsUrl}/bookingview/user/${id}`)
      .pipe(
        map((data) => data.map(item => this.adapter.adapt(item)))
      )
  }

}
\end{lstlisting}

\subsection{Widget Module}
Il widget module è un modulo Angular che ha la responsabilità di dichiarare ed esportare il WidgetComponent, e di registrare il widget nel WidgetService.
\begin{lstlisting}[caption={booking-view-widget.module.ts},label={booking_view_widget_module},style=javascriptCode]
@NgModule({
    declarations: [
        BookingViewComponent
    ],
    imports: [
        CommonModule,
        WidgetComponentsModule,
        GlobalizationModule,
        GridModule
    ],
    exports: [
        BookingViewComponent
    ]
})
export class BookingViewWidgetModule {
    public static forRoot(): ModuleWithProviders<BookingViewWidgetModule> {
        return {
            ngModule: BookingViewWidgetModule,
            providers: [
                {
                    provide: APP_INITIALIZER,
                    useFactory: BookingViewWidgetModule.registerWidgets,
                    multi: true,
                    deps: [WidgetComponentsService]
                }
            ]
        };
    }

    private static registerWidgets(widgetsService: WidgetComponentsService): () => Promise<any> {
        const result = (): Promise<any> => {
            return new Promise((resolve, reject) => {
                const myWidgetDescriptor: WidgetDescriptor = {
                    code: 'bookingViewWidgetCode',
                    shortText: 'Booking View',
                    longText: 'Questo e' il widget che ti permette di visualizzare alcune informazioni *migliorato*',
                    icon: 'fa fa-calendar-o',
                    group: 'It Group'
                };
                widgetsService.register(myWidgetDescriptor, BookingViewWidget, BookingViewComponent);

                resolve();
            });
        };
        return result;
    }
}
\end{lstlisting}

%\section{Struttura backend}
%Lo sviluppo di un widget precedentemente si divideva nelle seguenti fasi:
%\begin{itemize}
%\item sviluppo del Widget Component
%\item sviluppo di una classe Widget contenente la logica di business legata ai parametri
%\item creazione di un modulo per il widget e conseguente registrazione in [...]
%\item sviluppo di un servizio contenente la logica di ottenimento dati per il widget
%\item sviluppo della logica di ottenimento dati nell'applicativo backend
%\end{itemize}
