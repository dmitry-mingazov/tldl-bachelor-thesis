\chapter{Architettura legacy}
%In questo capitolo verrà discussa la struttura precedentemente utilizzata %nell'applicativo e [...]
%L'applicativo presentava una parte backend che aveva la responsabilità di 
%fornire degli endpoint REST attraverso i quali l'applicativo frontend potesse %ottenere i dati da far visualizzare all'utente finale attraverso i widget
\section{Widget}
Un widget è una piccola applicazione, semplice e immediata, che mostra all'utente dati e informazioni, di solito sotto forma di grafici o tabelle.
La sua struttura si articolava in diversi file con scopi diversi:
\begin{itemize}
\item \textbf{widget.component.html} contiene il codice HTML che descrive l'aspetto grafico che il widget avrà all'interno della dashboard, e rappresenta i dati contenuti dal widget.component.ts
\item \textbf{widget.component.ts} è un Component Angular, contiene i dati che il widget deve visualizzare e si occupa di ottenere i dati interfacciandosi con il servizio opportuno
\item \textbf{widget.ts} è la classe che gestisce la configurazione del widget
\item \textbf{widget.service.ts} è un servizio Angular, contiene la logica di accesso ai dati che il widget deve visualizzare
\item \textbf{widget.module.ts} è un modulo Angular, contiene la dichiarazione del componente widget e la sua registazione al WidgetService
\end{itemize}

\subsection{Widget Component}
Il widget

\section{Struttura backend}
Lo sviluppo di un widget precedentemente si divideva nelle seguenti fasi:
\begin{itemize}
\item sviluppo del Widget Component
\item sviluppo di una classe Widget contenente la logica di business legata ai parametri
\item creazione di un modulo per il widget e conseguente registrazione in [...]
\item sviluppo di un servizio contenente la logica di ottenimento dati per il widget
\item sviluppo della logica di ottenimento dati nell'applicativo backend
\end{itemize}
